\chapter{Introdução}
\label{chapter:introducao}

\section{Contexto e Motivação}

%A maioria dos idiomas é baseada em sons audíveis. Sendo assim, as pessoas estão naturalmente acostumadas à língua falada e, para a maioria delas, a comunicação diária é regida pela audição \cite{Duke2009}. No entanto, existem 466 milhões de pessoas, cerca de 6,1\% da população mundial, com perda auditiva incapacitante \cite{WHO2018}. Com isso, grande parte delas têm a necessidade de ``ouvir com os olhos'', algo possível por meio das línguas de sinais.

%As línguas de sinais são formas de comunicação essencialmente baseadas no sentido da visão. Assim, conversas e informações são transmitidas visualmente usando diferentes formas/movimentos das mãos. Para isso, seus usuários combinam movimentos articulados das mãos, expressões faciais, movimentos da cabeça e do corpo para comunicar sentimentos, intenções, humor, além de ideias complexas ou abstratas \cite{Duke2009}.

Segundo a Organização Mundial de Saúde (OMS), existem 466 milhões de pessoas, cerca de 6,1\% da população mundial, com perda auditiva incapacitante \cite{OMS2018}. Sendo assim, grande parte delas têm a necessidade de ``ouvir com os olhos'', algo possível por meio das línguas de sinais, as quais permitem que conversas e informações sejam transmitidas visualmente usando diferentes formas de comunicação sinalizadas. Para isso, seus usuários combinam movimentos articulados das mãos, expressões faciais e movimentos corporais para se comunicarem \cite{Duke2009, Quadros2019, Honora2017}.

De acordo com a Organização das Nações Unidas (ONU), tecnicamente não existe uma língua de sinais universal, principalmente devido às singularidades linguísticas em questão, as quais impõem barreiras contundentes para a universalidade das línguas de sinais \cite{ONU2019, Quadros2019}. Nesse contexto, a Federação Mundial de Surdos (WFD -- \textit{World Federation of the Deaf}) estima que existam aproximadamente 72 milhões de surdos em todo o mundo, os quais coletivamente utilizam mais de 300 línguas de sinais diferentes \cite{ONU2019}.

Um relatório da WFD relata que tamanha diversidade agrava ainda mais a falta de consistência em todo o mundo nos processos de interpretação e tradução das línguas de sinais, evidenciando uma linha de pesquisa relevante \cite{Napier2019}. De acordo com \citeonline{Duke2009}, caso todas as línguas de sinais fossem unificadas, elas representariam o terceiro idioma mais usado nos Estados Unidos e o quarto idioma mais usado no mundo.

No Brasil, a Língua Brasileira de Sinais (Libras) foi reconhecida oficialmente em 2002 \cite{Quadros2019, Honora2017}. Desde então, algumas iniciativas foram fomentadas em âmbito nacional para inclusão social/cultural de seus usuários. Segundo o último censo demográfico do Instituto Brasileiro de Geografia e Estatística (IBGE) \cite{IBGE2010}, existem cerca de 9,7 milhões de pessoas com limitações auditivas no Brasil, o que equivale à 5,1\% da população. Vale ressaltar que esses dados estatísticos não foram coletados visando um levantamento sobre o cenário nacional com relação à Libras. Todavia, tais pessoas são usuárias em potencial dessa língua de sinais, mas infelizmente apenas uma parcela delas terá acesso à alfabetização, tendo em vista as dificuldades socioeconômicas do país.

Por outro lado, o advento da tecnologia vem impactando significativamente o processo de ensino e aprendizagem das línguas de sinais. Com isso, soluções disruptivas, como o uso de dispositivos inteligentes ou técnicas de tradução simultânea, podem orientar o desenvolvimento de aplicações educacionais cada vez mais efetivas e acessíveis aos aprendizes das línguas de sinais \cite{Napier2019}.

Em uma perspectiva relacionada, as Tecnologias de Informação e Comunicação (TICs) têm alterado não apenas as interações pessoais, mas também as práticas de ensino. A globalização das TICs criou um contexto educacional sem precedentes: mais flexível, conectado e inteligente. As tecnologias portáteis, juntamente com as redes de computadores, tornam-se cada vez mais presentes na vida cotidiana, promovendo um acesso onipresente à informação \cite{Cilli2017}.

As TICs dizem respeito a processos e produtos que estão em constante transformação, provindos especialmente da eletrônica, da microeletrônica e da área de telecomunicações. O escopo de ação dessas tecnologias é virtual e utiliza a informação como o principal insumo \cite{Cilli2017}. Sendo assim, diferentes áreas de conhecimento tendem a se beneficiar com o uso dessas tecnologias, principalmente em domínios complexos como o da educação inclusiva baseada em línguas de sinais.

Em sentido amplo, educação indica o meio em que os hábitos, costumes e valores de um determinado grupo são transferidos de uma geração para outra, e estes vão se formando a partir de situações presenciadas e de experiências adquiridas no decorrer da vida. Em específico, educação é um processo contínuo de desenvolvimento das faculdades físicas, intelectuais e morais do ser humano para que este se integre da melhor forma possível em sua comunidade \cite{Cilli2017,Quadros2019}. Note que, em ambas as perspectivas, a inclusão social é essencial para que o indivíduo tenha acesso adequado à educação e isso independe de sua língua dominante (de sinais ou falada) \cite{Quadros2019}. 

A educação é um processo constante e evolutivo que visa a integração das pessoas à sociedade ou grupo ao qual pertencem \cite{Cilli2017,Quadros2019}. Logo, a inclusão social e digital são aspectos importantes no processo educacional moderno. Nessa perspectiva, as TICs podem contribuir para tornar as práticas educacionais mais inclusivas, personalizando o processo de ensino e aprendizagem ao contexto dos alunos, especialmente em cenários onde a diversidade linguística é essencial, como na educação bilíngue.

\section{Objetivos}

Tendo em vista o contexto e motivações apresentados, este projeto tem como principal objetivo prover uma infraestrutura, que orquestre recursos educacionais públicos visando a educação inclusiva para surdos. Ressalta-se que o conceito de inclusão, na identidade surda, traz consigo um ideal baseado na educação bilíngue. Sendo assim, a infraestrutura em questão permitirá o desenvolvimento de aplicações educacionais bilíngues. %Em especial, a arquitetura SOA foi escolhida devido a sua flexibilidade e compatibilidade com estilos arquiteturais já consolidados na Engenharia de Software (ES) e Industria. %Como por exemplo, \textbf{Re}presentational \textbf{S}tate \textbf{T}ransfer (REST) \cite{Fielding2000} ou Microserviços \cite{Sommerville2015}, os quais, inclusive, podem ser implementados juntos.
Com base nesse objetivo geral, são estabelecidos os seguintes objetivos específicos:

\begin{enumerate}
\item Identificação de um conjunto básico de especificações técnicas e requisitos de desenvolvimento relacionados à educação inclusiva por meio das línguas de sinais, com o apoio das TICs; %Obtido, principalmente, através de um Mapeamento Sistemático (MS);
\item Proposição de uma arquitetura que favoreça o desenvolvimento de soluções educacionais inclusivas no contexto das línguas de sinais. Desta forma, a infraestrutura resultante proverá todos os insumos necessários para o desenvolvimento de ferramentas de Tecnologia Assistiva (TA), as quais devem potencializar de forma significativa a participação do surdo no desempenho de suas tarefas educacionais.
\item Avaliação da arquitetura proposta por especialistas, tendo em vista o refinamento/evolução da abordagem em função de \textit{feedbacks} das áreas da ES, educação e línguas de sinais. Com isso, a arquitetura é devidamente instanciada e implantada, provendo uma infraestrutura de apoio à criação de soluções educacionais com suporte a línguas de sinais;
\item Desenvolvimento de uma aplicação educacional móvel, criada a partir da infraestrutura proposta, no contexto da educação bilíngue (Libras e Português). Para isso, um tópico educacional específico será definido tendo em vista o ensino e aprendizagem por meio das línguas em questão. %Nesse caso, a Matemática foi o tópico educacional escolhido por ser uma disciplina neutra, ou seja, que não favorece nenhuma das línguas (falada ou de sinais). 
Além disso, é importante ressaltar que as soluções geradas a partir dessa arquitetura/infraestrutura comum serão sensíveis ao contexto de seus usuários, obtendo informações importantes para o tratamento de eventuais regionalidades (presentes em ambas as línguas, falada e de sinais);
\item Condução de experimentos com aprendizes e tutores, %preferencialmente em uma sala de aula inclusiva, 
a fim de avaliar a TA criada e, indiretamente, sua respectiva arquitetura.
\end{enumerate}

Os objetivos apresentados buscam responder à seguinte questão de pesquisa: Como as TICs, aliadas a boas práticas de desenvolvimento e infraestrutura, podem auxiliar/facilitar a criação de aplicações educacionais inclusivas para surdos, por meio de uma interface pública e bilíngue?

%Para o desenvolvimento de ambientes educacionais efetivos, é essencial que sejam consideradas as características intrínsecas dos aprendizes, incluindo suas limitações físicas. Entretanto, usuários com deficiência auditiva ainda enfrentam dificuldades no processo de ensino devido a ambientes inadequados de aprendizagem, que geralmente não foram planejados/projetados pensando em usuários de línguas de sinais \cite{Snoddon2018}.

%Uma era disruptiva é evidente, na qual a disseminação de novas tecnologias e a interconectividade global têm potencial para acelerar o progresso humano, diminuir a barreira digital e desenvolver soluções cada vez mais inclusivas \cite{Itu2019a,Itu2019b}. Por esses motivos, é necessário refletir sobre o ensino e aprendizagem das línguas de sinais, considerando o crescente uso de tecnologias no campo educacional.

%De fato, as TICs podem facilitar o acesso ao conhecimento, especialmente no contexto de pessoas com algum tipo de deficiência. Nesse cenário, este trabalho apresenta a condução de um Mapeamento Sistemático (MS), a fim de destacar o estado da arte considerando o uso de tecnologias no ensino e aprendizagem de línguas de sinais.

%O MS selecionou 139 estudos primários através de uma abordagem de busca baseada em \textit{Quasi-Gold Standard} (QGS) \cite{Zhang2011}, obtendo uma visão geral sobre as principais soluções de software, hardware e contribuições teóricas relacionadas ao domínio em questão. Dessa forma, foi possível delinear os estudos selecionados com foco na Libras. Considerando o cenário nacional, a Libras está presente em 16 destes trabalhos, sendo superada apenas pela \textit{American Sign Language} (ASL), e representando cerca de 11,5\% dos estudos primários do MS. De forma complementar, outros 46 estudos, selecionados manualmente em eventos nacionais, foram incorporados aos estudos primários com a intenção de expandir os resultados e discussões no âmbito da Libras.

\section{Organização}

Neste capítulo foi apresentado o contexto geral ao qual este trabalho de doutorado está inserido, além das motivações e justificativas para o seu desenvolvimento e os objetivos almejados para sua conclusão. 

O Capítulo \ref{chapter:fundamentacao-teorica} apresenta a fundamentação teórica que embasa este trabalho, abordando suas temáticas principais: línguas de sinais, educação e tecnologia, com ênfase na educação inclusiva e tecnologia assistiva.

O Capítulo \ref{chapter:mapeamento-sistematico} apresenta um mapeamento sistemático sobre o cenário atual de pesquisas envolvendo o ensino e aprendizagem com línguas de sinais por meio do uso de tecnologias. Desta forma, foi possível analisar os cenários global e nacional no que tange o escopo de pesquisa em questão.

O Capítulo \ref{chapter:proposta} apresenta a proposta deste trabalho de doutorado, retomando a caracterização da pesquisa e apresentando as atividades previstas, cronograma, procedimentos metodológicos, resultados esperados e publicações. 