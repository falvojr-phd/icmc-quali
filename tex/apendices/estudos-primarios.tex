\chapter{Síntese dos estudos discutidos nos âmbitos nacional e internacional}
\label{apendice:estudos-primarios}

\begin{small}
\setlength\tabcolsep{0.1cm}
\begin{longtable}{|c|C{2.75cm}|c|L{10cm}|}
\hline
\textbf{Estudo} & \textbf{Tipo de Solução} & \textbf{Avatar} & \textbf{Síntese} \\ \hline
%\textit{BRA2} & Mobile & Não & Jogo baseado em mensagens instantâneas, chamado \textit{LibrasZap}. \\ \hline
\textit{BRA6} & Mobile & Sim & Avaliação das soluções \textit{Hand Talk}, \textit{ProDeaf} e \textit{Rybená}, com foco na acessibilidade linguística para crianças surdas. \\ \hline
\textit{BRA7} & Mobile & Sim & Avaliação dos aplicativos \textit{Hand Talk} e \textit{ProDeaf}, com base na teoria da aprendizagem multimídia e em heurísticas de usabilidade. \\ \hline
\textit{BRA8} & Mobile & Sim & Avaliação dos aplicativos \textit{Hand Talk} e \textit{ProDeaf}, considerando os desafios relacionados à desambiguação. \\ \hline
\textit{BRA10} & Mobile & Sim & Avaliação dos aplicativos \textit{Hand Talk} e \textit{ProDeaf}, ponderando a incidência de traduções baseadas em datilografia. \\ \hline
\textit{BRA11} & Mobile & Sim & Avaliação dos aplicativos \textit{Hand Talk} e \textit{ProDeaf}, sob a perspectiva de surdos e ouvintes no curso de formação continuada de professores. \\ \hline
\textit{BRA12} & Web & Não & Plataforma educacional intitulada \textit{SalaBil}, com ênfase na educação bilíngue (Português e LIBRAS). \\ \hline
\textit{BRA16} & Mobile & Sim & Avaliação dos aplicativos \textit{Hand Talk} e \textit{ProDeaf}, considerando a análise de tradução automática Português-LIBRAS no contexto do edital do ENEM. \\ \hline
\textit{BRA18} & Mobile & Sim & Avaliação dos aplicativos \textit{Hand Talk} e \textit{ProDeaf}, sob três perspectivas: multimidiática; multimodalidade; e quanto à socialização do sujeito surdo. \\ \hline
\textit{BRA22} & Framework & Não & \textit{Framework CAP 1.0}, que visa a criação e uso de arquiteturas pedagógicas. \\ \hline
\textit{BRA23} & Desktop & Não & Jogo idealizado para instigar o pensamento lógico de crianças surdas através de noções iniciais de robótica, denominado \textit{LibrasBot}. \\ \hline
%\textit{BRA24} & Desktop & Não & A \textit{ForcaBRAS} é uma proposta de jogo para o ensino lúdico e intuitivo da representação do alfabeto e números usando LIBRAS. \\ \hline
%\textit{BRA25} & RA & Não & Jogo da memória baseado em RA que visa apoiar usuários da LIBRAS no processo de aprendizagem. \\ \hline
%\textit{BRA27} & Framework & Não & Apresentação da \textit{CAP-APL}, uma plataforma Web que disponibiliza recursos digitais através de APs (Português e LIBRAS). \\ \hline
%\textit{BRA29} & RV & Sim & Dicionário temático visual-gestual baseado em RV para LIBRAS. \\ \hline
\textit{BRA31} & RA/RV & Não & O jogo \textit{LibrAR} utiliza RA e RV para auxiliar no ensino do alfabeto e de algarismos numéricos em LIBRAS. \\ \hline
%\textit{BRA33} & Mobile & Sim & Apresentação do aplicativo \textit{MIDOAA}, que embarca objetos de aprendizagem acessíveis para computação. \\ \hline
%\textit{BRA34} & Mobile & Não & Jogo didático multidisciplinar para alunos surdos da Educação de Jovens e Adultos, chamado \textit{Quiz ClassRoom LIBRAS}. \\ \hline
%\textit{BRA35} & Framework & Não & Plataforma \textit{Construtor de Arquiteturas Pedagógicas (CAP)} para a aprendizagem de LIBRAS. \\ \hline
%\textit{BRA36} & Mobile & Não & Aplicativo educacional móvel baseado em técnicas de gamificação que tem o objetivo de apoiar o ensino-aprendizagem da LIBRAS. \\ \hline
%\textit{BRA37} & Proposta de Design & Não & Modelo para educação inclusiva baseado na semiótica peirceana. \\ \hline
\textit{BRA39} & Web & Não & Jogo \textit{Gestus}, que tem por objetivo apresentar a Libras à crianças ouvintes, com idades entre 7 a 10 anos, de forma lúdica/humanizada. \\ \hline
%\textit{INT17} & Tradução de Máquina & Sim & Processamento de legendas para geração automática de linguagem de sinais. \\ \hline
%\textit{INT34} & RV & Sim & Ambiente de RV com suporte a chats para deficientes auditivos e para o ensino de LIBRAS. \\ \hline
\textit{INT47} & Tradução de Máquina & Sim & Solução baseada em tradução de máquina com suporte a avatares 3D, além de apresentar formalmente a concepção de um corpus com suporte a LIBRAS. \\ \hline
\textit{INT53} & Desktop & Não & Solução embarcada em DVD para o ensino da Língua de Sinais Australiana (Auslan) com possibilidade de configurações regionais. \\ \hline
\textit{INT56} & API & Não & Apresenta a arquitetura BDC-API, que traduz o material didático digital de MOOCs para surdos e cegos. \\ \hline
\textit{INT73} & Web & Não & \textit{SignQuiz}, uma aplicação educacional Web para ensino da Língua de Sinais Indiana (ISL) através de uma técnica de reconhecimento automático de sinais. \\ \hline
\textit{INT79} & Proposta de Design & Não & Proposta inicial de trabalho e metodologia que discorre sobre a criação de uma língua de sinais unificada, visando reduzir as barreiras sociais enfrentadas pelos surdos atualmente. \\ \hline
%\textit{INT80} & Proposta de Design & Sim & Desenvolvimento de um avatar intérprete de LIBRAS. \\ \hline
%\textit{INT92} & Mobile & Não & Jogo de adivinhação multiplayer para aprendizagem e prática da LIBRAS. \\ \hline
\textit{INT86} & Legendas com Línguas de Sinais & Sim & Criação de legendas automáticas em línguas de sinais para recursos educacionais online, com o objetivo de torná-los acessíveis à comunidade de deficientes auditivos. \\ \hline
\textit{INT88} & Óculos Inteligentes & Não & Uso do conceito de RA através de óculos inteligentes no apoio a pessoas com deficiência auditiva. \\ \hline
\textit{INT109} & Desktop & Não & Jogo educacional para o aprendizado de operações matemáticas e números em LIBRAS, denominado \textit{MatLIBRAS Racing}. \\ \hline
\end{longtable}
\end{small}
